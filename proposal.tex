\documentclass[dvips,12pt]{article}

% Any percent sign marks a comment to the end of the line

% Every latex document starts with a documentclass declaration like this
% The option dvips allows for graphics, 12pt is the font size, and article
%   is the style

\usepackage[pdftex]{graphicx}
\usepackage{url}

% These are additional packages for "pdflatex", graphics, and to include
% hyperlinks inside a document.

\setlength{\oddsidemargin}{0.25in}
\setlength{\textwidth}{6.5in}
\setlength{\topmargin}{0in}
\setlength{\textheight}{8.5in}

% These force using more of the margins that is the default style

\begin{document}

% Everything after this becomes content
% Replace the text between curly brackets with your own

\title{Implementation of (Somewhat) Fully Homomorphic Encryption on a Field Programmable Gate Array}
\author{Joseph Kirk}
\date{\today}

% You can leave out "date" and it will be added automatically for today
% You can change the "\today" date to any text you like


\maketitle

% This command causes the title to be created in the document

\section{Motivation}

% An article style is separated into sections and subsections with 
%   markup such as this.  Use \section*{Principles} for unnumbered sections.

As more and more organisations turn to cloud-based computing for their business needs, 
sensitive data sees more opportunity for breaches in privacy than ever before. For example,
computation providers allow other companies to make use of their hardware remotely for
more intense computations than the companies would otherwise be able to perform; however,
any data computed on must be provided in an unencrypted form, disallowing the use of
this service on private data. Homomorphic encryption aims to allow computation on encrypted
data, with computations performed on ciphertext being reflected in the plaintext, thereby allowing
for computation providers to work with sensitive data without being able to decrypt it. Furthermore,
homomorphic cryptosystems allow for the evaluation of encrypted functions on encrypted data,
useful where algorithms themselves may be considered sensitive (e.g. in MRI machines).
\par Although the theoretical usefulness of fully homomorphic encryption is high, the practicality of it is less so.
Computations on ciphertext require exponentially more time than the counterpart operation on 
plaintext, making real-time computation with FHE impossible until advances are made in
increasing performance. This proposal considers the various approaches to making a practical
implementation of homomorphic encryption a reality.

\section{Background}

In order to explore the background of fully homomorphic encryption, the terms 
somewhat homomorphic and partially homomorphic encryption must be established.
The distinction is as follows:
\begin{description}
\item[Partially homomorphic encryption] refers to encryption schemes with homomorphic
properties in either addition or multiplication. Without homomorphism in both operations,
these encryption schemes are severely limited in terms of functions they are able to evaluate.
\item[Somewhat homomorphic encryption] refers to encryption schemes with homomorphic
properties in both addition and multiplication, but with error vectors that increase with each
subsequent addition or multiplication. This limits the scheme to evaluating only simple functions,
as too large an error vector prevents accurate decryption.
\end{description}
\par Fully homomorphic encryption expands upon somewhat homomorphic encryption by introducing
methods of reducing error vector blow-up. By controlling for the errors in somewhat homomorphic
encryption schemes, fully homomorphic encryption allows for the evaluation of arbitrarily complex
functions. 

\par The first fully homomorphic encryption scheme was proposed by Craig Gentry.\cite{Gentry09} 
The scheme used a polynomial ring and ideal lattices in order to establish homomorphism in both
addition and multiplication, and introduced the concept of \emph{bootstrapping} to prevent the
error vector from blowing up. Bootstrapping allows for the scheme's Eval function, which evaluates
arbitrary circuits on a given ciphertext homomorphically, to evaluate its own decryption circuit; in doing so,
it allows for a given ciphertext to be \emph{recrypted} into a ciphertext with a smaller error vector. \texttt{[Deepayan:
Sorry for the lacking explanation - I don't fully understand how this works yet.]}
\par As of yet, however, bootstrapping is considered prohibitive. The original scheme, as implemented
by Gentry and Halevi, is stated by the authors to be practically secure with $2^{13}$ or $2^{15}$ dimensions;
the recryption process for a scheme with these dimensions takes 2.8 min or 31 min, respectively. Furthermore, 
the respective public key sizes for the schemes of these sizes are 284MB and 2.25GB, respectively.\cite{gentryimp}
Clearly such a system is far from practical in lightweight or time-sensitive operations. For this reason, research into homomorphic
encryption has divided into two branches with similar goals yet different approaches. Some research attempts to implement
bootstrapping in ever more efficient and lightweight forms, while other research focuses instead on improving the
operation limits of somewhat homomorphic encryption schemes in other ways, such as optimising operations to reduce the increase in noise.

\par The original scheme proposed by Gentry is based upon ideal lattices; this gives it post-quantum security but 
increases the complexity of using the system to prohibitive levels for non-experts. Improvements have been made to the ideal lattice FHE scheme by
Gentry and others, such as speeding up the \texttt{KeyGen} function\cite{gentryimp} or reducing the size of the keyspace without sacrificing
security.\cite{smartvercauteren} An additional three FHE scheme archetypes are found in research:
\begin{description}
\item[Over Integers FHE] may be regarded as the simplest scheme to understand from a non-expert perspective. First proposed by Van Dijk et al.,\cite{vd} it encrypts a message as follows:
\begin{equation}
E(m) = m + 2r + pq;
\end{equation}
where $p$ and $q$ are large primes, and $r$ is a random number much smaller than $p$. In this form, $p$ is a symmetric secret key, and $c$ may be decrypted as:
\begin{equation}
D(c) = (c \bmod p) \bmod 2.
\end{equation}
It may be plainly seen from the above equations that the scheme is bounded. In order to deal with this, Van Dijk et al. introduced bootstrapping, much in the same manner as Gentry. 

\end{description}

\bibliographystyle{IEEEtran}
\bibliography{IEEEabrv,homo}
\end{document}
